\documentclass[letterpaper]{article}

\author{William Brown (whb2107)}

\usepackage{amsmath,amssymb,appendix,color,
  graphicx,index,listings,lscape,natbib,url,wasysym}

% font definitions
\usepackage{mathpazo} % math & rm
\linespread{1.05}        % Palatino needs more leading (space between lines)
\usepackage[scaled]{helvet} % ss
\usepackage{eulervm}
\normalfont
\usepackage[T1]{fontenc}

\definecolor{gray}{gray}{0.95}

\renewcommand{\familydefault}{\rmdefault}

\lstset{
  language=Python,
  basicstyle=\footnotesize\ttfamily,
  numbersep=5pt,              
  tabsize=2,                  
  extendedchars=true,         
  breaklines=true,            
  stringstyle=\ttfamily, 
  showspaces=false,      
  showtabs=false,        
  xleftmargin=17pt,
  framexleftmargin=17pt,
  framexrightmargin=5pt,
  framexbottommargin=4pt,
  backgroundcolor=\color{gray},
  showstringspaces=false 
 }

\newcommand\imgfig[4]{
\begin{figure}[h]
  \centering
  \includegraphics[scale=#2]{figures/#1}
  \caption{#3}
  \label{fig:#4}
\end{figure}}

\newcommand\figref[1]{Figure \ref{fig:#1}}
\newcommand\tabref[1]{Table \ref{tab:#1}}
\newcommand\code[1]{\texttt{#1}}
\newcommand\pipe[0]{\;|\;}
\newcommand\st[0]{\;\mathrm{s.t.}\;}
\newcommand\func[3]{#1 \,:\, #2 \longrightarrow #3}
\newcommand\relsol[1]{\frac{ #1 }{ #1_{\astrosun} }}
\newcommand\sol[1]{ #1_{\astrosun}}

\newindex{todo}{tod}{tnd}{TODO List}

\newcommand\todo[1]{
  % Add to todo list
  \index[todo]{#1}
  % Make the margin par
  \marginpar{
    \raggedright
    \textbf{TODO}:\\
    #1
  }
}

\newcommand\todolist{\printindex[todo]}

\setcounter{secnumdepth}{5}
\setcounter{tocdepth}{5}

\newcommand\startappendix{\appendix\appendixpage\addappheadtotoc}

\title{annotate23d: Human-friendly 3D modelling}
\begin{document}
\maketitle
\section{Modelling Primitives}
Primitives are split into two types: cylinderoids and ellipsoids.  Cylinderoids
have significantly more features than ellipsoids, in part due to the
completeness of their implementation, and in part because they are more powerful
primitives.
\subsection{Cylinderoids}
Cylinderoids are created by drawing a spine on the screen. This spine is made up
of every point at which a touch was detected during the swipe gesture --
usually, this means that there is no more than about one pixel between each
point. This overdensity of the spine is unnecessary, so, on constructing the
cylinderoid, the spine is resampled such that the spine points are no closer
than 30 pixels away from each other. If resampling would cause the spine to have
fewer than 3 control points, that target distance is halved and it attempts to
resample again. This is repeated until a good resampling is found.

All cylinderoids are defined by their spine points, their radius at each spine
point, and the tilt at each spine point. However, there are a number of
annotations that can be applied to each cylinderoid that effect all of these
parameters (location, radius and spine). These are stored in fields in the
Cylinderoid class. Then, when the mesh is computed, the class computes the
locations, radii and tilts with all of the constraints satisfied, and uses those
to generate the mesh.

Tilts are slightly different from the other two parameters because they are not
guaranteed to be defined everywhere over the spine. While each spine point has a
location and a radius, it's possible that the user has defined the tilt in some
places, or nowhere at all. \verb#NaN# is stored in the tilt vector where the
user has not specified a tilt; the app then linearly interpolates between tilts
along the spine, or, if no tilt is assigned anywhere, it sets the tilt to zero
along the whole spine.

The mesh of a cylinderoid is generated by sweeping a segmented circular ring
along the spine, and connecting each ring with triangles. Each ring is defined
by a location (\verb#spinePoint# in \verb#Cylinderoid::generateMesh#), a tangent
vector (mistakenly named \verb#derivative#) and a radius that is perpendicular
to the tangent of the spine (\verb#radius#).

\subsection{Ellipsoids}
Ellipsoids are significantly less advanced than cylinderoids, because they do
not support any annotations and have significantly fewer degrees of freedom.
Each ellipsoid has a center of mass \verb#com#, a major axis length \verb#a#, a
minor axis length \verb#b# and an angle between the major axis and the x-axis,
\verb#phi#. The boundary of the ellipsoid in two dimensions in then defined as
\begin{equation*}
b = c + (\cos t) a 
	\left(\begin{smallmatrix}\cos \phi\\ \sin \phi\end{smallmatrix} \right)
	+ (\sin t) b
	\left(\begin{smallmatrix}\sin \phi\\ -\cos \phi\end{smallmatrix} \right)
\end{equation*}
Where $b$ is a boundary point, $c$ is the center of mass and $t$ varies from 0
to $2\pi$. The mesh of an ellipsoid is constructed using the same method as that
of the cylinderoids, but the radius of each ring is determined using $a$ and $b$
instead of the user-provided radii.

\subsection{Annotations}
Six annotations are available to the user: same-length, same-scale, same-tilt,
connection, alignment and mirroring. The annotation objects themselves, defined
in \verb#Annotations.h#, only provide methods for calculating offsets and the
like; they do not generate meshes. The Cylinderoid object itself uses the
information in the annotations to generate its mesh.

\subsubsection{Same-length}
Same length annotations ask the user to pick two cylinderoids, and then sets the
length of those two cylinderoids equal to the mean of their original lengths.
The transformation is taken by translating the cylinder's center of mass to the
origin, scaling the spine by the required amount, and then translating back to
its original position. Note that the radii are uneffected, so this is a scaling
of the spine, not the shape as a whole.

\subsubsection{Same-scale}
Same scale annotations ask the user to pick two spine points (which can
optionally be on the same cylinderoid) and then sets the radii at those spine
points to be the mean of the original radii. However, this can often result in
strange-looking discontinuities in the radii, so a weak smoothing operator is
applied to the radii before drawing.

\subsubsection{Same-tilt}
Same tilt annotations ask the user to pick two spine points with associated tilt
and then sets the tilt at those spine points to be the mean of the original
tilts. As the tilt is interpolated over the spine and, in most situations, is
very sparsely defined over the spine, no smoothing is necessary here.

\subsubsection{Connection}
Connection annotations take two shapes -- one stationary, and one connector --
and an intersection point. The meshes for the two shapes are then generated
using all annotations and those meshes are intersected with a ray that is
emitted orthogonally from the image plane. If the intersection with the
stationary object is $z_s$ from the image plane, and the connector intersection
is $z_c$ away, the connector is translated $z_s - z_c$ units along the z-axis,
where the z-axis points into the screen.

\subsubsection{Alignment}
Alignment annotations take one shape which must be the connector in a connection
annotation. They find the intersection point between the ray and the connector,
and then translate that point to be on the symmetry sheet of the stationary
object.

\subsubsection{Mirror}
Mirror annotations take one shape which also must be the connector in a
connection annotation, and they create a duplicate mesh reflected about the
symmetry plane of the stationary object. In order to do this, the full mesh is
generated for the connector. We then copy the mesh, subtracting two times the
projection of each vertex onto the symmetry plane normal from each vertex. This
is then unioned with the global mesh to produce a new global mesh.

\section{App architecture}
\subsection{User interaction}
The parent file of the entire app is \verb#WorkspaceViewController#, which is
responsible for initializing all other parts of the app, as well as the layout
of the main view. The layout is defined using a NIB file.

\verb#WorkspaceViewController# also manages what tool is currently active, and
acts as a delegate for all touch events. It receives events via the
\verb#UIResponder# protocol (\verb#touchesBegan, touchesMoved, touchesEnded#)
and either delegates to \verb#WorkspaceUIView# or \verb#DrawPreviewUIView# or
uses them to manipulate the view (zoom or pan).

\verb#DrawPreviewUIView# is used when creating new cylinderoids or ellipsoids.
It is reponsible for reading all touch positions and recording them, and when
the touch finishes it passes an array of points back up to
\verb#WorkspaceViewController#, which creates the new primitive and stores it in
the workspace.

The workspace is a \verb#WorkspaceUIView#, and it's responsible for most of the
interesting interaction between user and app. It keeps track of what shapes
exist in the workspace, and also handles all touch events related to annotations
or manipulating existing shapes. All of the stories for creating new annotations
are defined here, in methods whose names match the annotation name. It also
draws unselected shapes.

If at any time a shape is selected, the workspace stores a reference to a
\verb#ShapeTransformer# for that shape. Shape transformers are responsible for
all of the handles associated with an object. They are also responsible for
drawing selected objects -- this includes handles and annotations. Most
annotations are handled by calling class methods in \verb#AnnotationArtist#.

The \verb#EllipsoidTransformer# class is very simple; it handles
translate-rotate-scale (TRS) for ellipsoids, and handles manipulation of the
major and minor axis handles. These handles have a simple interaction model; as
the user moves the handle, the transformer finds the projection of the user's
touch on the axis which they are manipulating, and sets that dimension to that
projection.

The \verb#CylinderoidTransformer# class is much more complex because of the
increased complexity of cylinderoids. It handles TRS for the shape as a whole,
but the user can also select a spine handle and move that handle, or pinch to
resize the radius at that point, or tap again and drag to change the tilt at
that point. All of this is handled in \verb#CylinderoidTransformer#.

\subsection{Mesh generation}
When the user requests a render or exports the model to an OBJ file,
\verb#MeshGenerator# \verb#::globalMesh# is invoked on the workspace. This method
requests meshes from all of the primitives defined in the workspace, and then
uses \verb#Mesh::combine# to combine all of these meshes into one single mesh.

The mesh format is unfortunately complex, but it is what's required by OpenGL ES
due to its lack of immediate mode. Each mesh object is an array of floats, where
each vertex is defined as six numbers: the $x,\,y\,z$ of its location, and the
$x,\,y\,z$ components of the normal for that vertex. Each triangle is a block of
three of these vertex definitions; each triangle is therefore a block of 18
floats which map to three verteces as described above.

This mesh is then passed to \verb#GlkRenderViewController#, which creates a
modal popover containing an OpenGL view, compiles the vertex and fragment
shaders and draws the mesh to the view.

Wavefront OBJ file generation is handled by \verb#ObjExporter#, which uses the
global mesh from \verb#MeshGenerator# to create a file in the iPad's local
storage that contains an OBJ file. Most of the complexity in this file comes
from the need to merge coincident verteces; since the mesh data structure
doesn't have any adjacency information, a naively-generated mesh file would be a
bag of triangles without adjacency. This would be useless for most graphics
algorithms and would result in significant wasted storage space. Verteces are
merged using their position to compute adjacency.

\section{Significant completed features}
\begin{itemize}
\item Creation and manipulation of cylinderoids and ellipsoids in 2D, including
least-squares fitting of drawn ellipses
\item Same-length, same-scale, same-tilt, connection, alignment and mirror
annotations
\item 3D mesh generation of all primitives, taking into account all annotations
\item OBJ file exporting via email
\item Background images using iOS's built-in photo browser
\item Interactive tutorial that informs the user what results their actions will
have
\end{itemize}

\section{Future work}
A list of things that were not completed that were discussed, or that I wish had
been done:
\begin{itemize}
\item Connections on ellipsoids. This is easy enough, and much of the
infrastructure is already there. It was not completed due to lack of time.
\item Manipulation of the OpenGL view and OpenGL preview picture-in-picture.
\item Cross-section shape adjustment
\item Better UI for manipulating annotations
\item Symmetry sheet manipulation
\end{itemize}
\end{document}
